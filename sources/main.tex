\documentclass{article}
\usepackage[utf8]{inputenc}

\usepackage[T2A]{fontenc}
\usepackage[utf8]{inputenc}
\usepackage[russian]{babel}

\usepackage{multienum}
\usepackage{geometry}
\usepackage{hyperref}

\geometry{
    left=1cm,right=1cm,
    top=2cm,bottom=2cm
}

\title{История}
\author{Лисид Лаконский}
\date{February 2023}

\newtheorem{definition}{Определение}

\begin{document}
\raggedright

\maketitle
\tableofcontents
\pagebreak

\section{Практическое занятие №6, «россия на путях „европеизации” в 18 веке»}

\subsection{Общие тенденции изменений в социальном и правовом статусе основны сословий российского общества в 18 веке}

В России со 2-й половины XVIII века утвердилось сословное деление на:

\begin{multienumerate}
    \mitemx{Дворянство, которое делилось на потомственное и личное}
    \mitemxxx{Духовенство}{Сословие почётных граждан}{Купечество}
    \mitemxxx{Разночинцев}{Мещанство}{Казачество}
    \mitemx{Крестьянство, которое делилось на лично свободных однодворцев и черносошных крестьян, а также на зависимых от феодалов удельных и крепостных крестьян}
\end{multienumerate}

С введением Петром I \textbf{«Табели о рангах»} стало возможным получение дворянства недворянами; для этого было достаточно получить чин низшего, XIV класса. С целью сдержать массовый приток недворян, с \textbf{1856 года} планка повышается до IX класса. С той же целью в \textbf{1832 году} вводится сословие почётных граждан, получавших ряд дворянских привилегий (в частности, свобода от телесных наказаний), но, вместе с тем, не получавших даже личного (не говоря уж о потомственном) дворянства.

Русское православное духовенство традиционно разделялось на \textbf{белое} (приходское), и \textbf{чёрное} (монашествующее).

Купечество также пользовалось рядом привилегий, и с конца XVIII века было разделено на \textbf{три гильдии}, членство в которых определялось размером капитала.

Те, кто не относился ни к одному сословию, назывались \textbf{разночинцы}. К ним относились российские подданные, положение которых было неопределённым; в частности, дети личных (не потомственных) дворян.

Русское крестьянство в сословной системе разделялось на ряд категорий:
\begin{enumerate}
    \item \textbf{Государственные крестьяне}, проживавшие на землях, принадлежавших государству
    \item \textbf{Помещичьи крестьяне}
    \item \textbf{Удельные крестьяне}, проживавшие на землях, принадлежавших императорской фамилии
    \item \textbf{Поссесионные} (приписные крестьяне), приписанные к определённым заводам
    \item \textbf{Однодворцы}
\end{enumerate}

\subsection{Усиление позиций дворянства в политической и экономической жизни страны}

На рубеже XVII – XVIII веков в России начались радикальные и крупномасштабные преобразования, инициированные Петром I. В стране происходили грандиозные перемены в самых разных областях жизни. \textbf{Дворянство в ходе этих реформ выступало в качестве основной силы}, на которую опиралась царская власть для достижения поставленных Петром I целей. Реформы, как правило, проводились таким образом, \textbf{чтобы дворянство в ходе их проведения ещё более укрепляло своё социально-экономическое положение и политическую власть}.

\hfill

В XVIII в. дворянство превратилось в сильное, сплочённое, ясно осознающее свои общие политические интересы, привилегированное сословие. Этому способствовала отмена местничества и другие преобразования Петровской эпохи. Особую роль для дворянства имела \textbf{Табель о рангах, вступившая в силу в 1722 г.} Этот документ ликвидировал старинную систему местничества при прохождении государственной службы. Титул и звание становились результатом продвижения по служебной лестнице. В Табеле о рангах приводился перечень гражданских, придворных и военных чинов (морских, сухопутных, артиллерийских, гвардейских). Всего устанавливалось 14 классов чинов. Достигнув чина определенного класса, военный или чиновник, не входивший в дворянское сословие, мог получить личное (14–9-й классы) или потомственное (с 8-го класса) дворянство. Все \textbf{высшие руководящие должности в государственном аппарате занимать могли только дворяне}.

\hfill

\textbf{26 января 1718} Пётр I издал \textbf{Указ о подушной переписи}. Этим указом ещё более \textbf{закреплялось привилегированное положение дворянства}. Оно \textbf{освобождалось от налогов и провозглашалось неподатным сословием}. Отсюда, кстати, вошло в русский язык и выражение подлец, подлый люд. Так дворяне противопоставляли своё благородное сословие тем неблагородным, которым подлежало выплачивать царёвы подати и налоги. В целом ряде законодательных актов, издававшихся при Петре I, Анне, Елизавете, Петре III и Екатерине II устанавливалось и закреплялось \textbf{исключительное монопольное право дворян на владение земельными угодьями на правах собственности}. С учётом уже существовавших на протяжении нескольких столетий крепостнических порядков, запрещавших крестьянам покидать свои места жительства без разрешения помещика, эти указы и другие законодательные акты фактически даровали дворянам не только права землевладельцев, но и право на крепостных крестьян. Это \textbf{сближало их положение с положением классических рабовладельцев}. Они так же \textbf{получили право покупать и продавать крестьян}.

\hfill

\textbf{Указ Петра I 1714 г. о единонаследии фактически уравнял боярское вотчинное землевладение и дворянское помещичье}. Но одновременно он вводил систему майората. Когда \textbf{всё недвижимое имущество и земля могло передаваться по наследству только одному наследнику}. Такая система со средневековой эпохи практиковалась в Англии и в некоторых других странах. Пётр I \textbf{стремился не допустить дробления дворянских землевладений}.

\hfill

Введение майората при наследовании вызвало упорное сопротивление и недовольство дворянства. И \textbf{в 1731 г. в начале своего правления императрица Анна, стремясь укрепить свою поддержку со стороны дворянства, правило майората отменила}. При её вступлении на престол высшие придворные сановники пытались ограничить самодержавную власть монарха. После смерти от оспы, находившегося на престоле 14-летнего внука Петра Великого Петра II высшие сановники, объединившиеся в Верховном тайном совете, предложили взойти на престол племяннице Петра I \textbf{Анне Иоанновне}, вдове герцога Курляндского, взяв с неё обещание ограничить права монарха и поделиться ими с Верховным тайным советом. Нечто подобное произошло в этот период в Швеции, заложив основы переходу там к конституционной монархии. Но русская история пошла по иному пути. Вступив на престол и заручившись поддержкой дворянства, императрица Анна официально объявила, что не намерена выполнять данные ею «верховникам» обещания и будет строго придерживаться самодержавного способа правления. \textbf{Верховный тайный совет был распущен, его члены были подвергнуты опале}. А роль дворянства, в особенности столичных офицеров гвардии ещё более усилилась. На протяжении всего XVIII в. именно \textbf{позиция столичных гвардейцев имела решающее значение в кризисных ситуациях, когда решался вопрос о престолонаследии}. Политический строй, сложившийся в России к XVIII в., порой называют \textbf{самодержавием, ограниченным дворцовыми переворотами}.

\subsubsection{Фактическое освобождение дворянства от государственной службы к концу XVIII в}

При тех порядках, что были установлены в царствование Петра I, служба для дворян была, если не каторгой, то весьма трудным и нелёгким делом, часто в разъездах вдали от родных мест. На этой службе запросто можно было потерять и жизнь и здоровье. И комфортные бытовые условия при прохождении службы гарантированы были не всегда, а в зависимости от множества обстоятельств. Царёва служба считалась обязательной и пожизненной. \textbf{В 1730 г. при императрице Анне условия службы дворян несколько облегчились. Срок службы ограничен до 25 лет. Начинать службу теперь можно было не в 15, а в 20 лет}. Один из мужчин в дворянской семье получил право вовсе не служить и заниматься общим семейным хозяйством.

\hfill

Но подлинный прорыв в закреплении дворянских привилегий и сведения к минимуму их обязанностей произошёл в \textbf{1762 г}. Император Пётр III, остававшийся на престоле лишь около одного года, успел издать манифест \textbf{«О даровании вольности и свободы всему российскому дворянству»}. Согласно ему \textbf{все дворяне вовсе освобождались от обязательной военной и государственной службы}. Она становилась для них делом добровольным. Этот манифест, стал по мнению многих историков, одной из причин Пугачёвского бунта, по сути дела гражданской войны конца XVIII в. Хотя в историографии и не принято её называть таким термином, чаще называют крестьянской войной 1773-1775 гг.

\hfill

До этого манифеста можно было говорить об определённой логике крепостнических помещичьих порядков. Крестьяне служат помещику, обеспечивают его всем необходимым, чтобы тот мог добросовестно служить царю и государству. Манифест 1762 г. резко ломал эту логику. Крестьяне не понимали, почему теперь они должны служить помещику, если теперь он мог становиться по сути дела бездельником.

\hfill

\textbf{При Екатерине II Пугачёвское восстание было жестоко подавлено}. Между тем до Великой Французской революции 1789 г. Екатерина II проводила политику \textbf{просвещённого абсолютизма}. Она исходила из идеи о том, что постепенно права и свободы людей должны расширяться. Поскольку только свободный человек способен проявить себя в творческой созидательной деятельности в полной мере. Но необходимое условие для этого – \textbf{просвещение и образование}. Иначе права и свободы могут быть использованы во зло. А наиболее просвещённым сословием в России в тот момент были дворяне.

\hfill

\textbf{В 1785 г. Екатерина II издала Жалованную грамоту дворянству}. В этом документе императрица \textbf{подтвердила свободу дворян от обязательной службы, уплаты налогов и выполнения различных повинностей}. К дворянам \textbf{запрещалось применять телесные наказания}. Были узаконены официальные   сословные организации — уездные и губернские дворянские собрания, формировавшиеся на выборной основе.

\subsection{Новая политическая сила — гвардия — и ее роль в решении судеб престолонаследия}

Создавая в \textbf{1692 году} гвардию, Петр хотел противопоставить ее стрельцам – привилегированным пехотным полкам московских царей, которые к концу XVII века стали вмешиваться в политику. Гвардия – первое и, может быть, наиболее совершенное создание Петра. Эти два полка – шесть тысяч штыков – по боевой выучке и воинскому духу могли потягаться с лучшими полками Европы. \textbf{Гвардия для Петра была опорой в борьбе за власть и в удержании власти}. По свидетельству современников, Петр часто говорил, что между гвардейцами нет ни одного, которому бы он смело не решился поручить свою жизнь. Гвардия для Петра была «кузницей кадров». Гвардейские офицеры и сержанты \textbf{выполняли любые поручения царя} – от организации горной промышленности до контроля за действиями высшего Генералитета. Гвардия всегда знала свой долг – была так воспитана. Она казалась Петру той идеальной моделью, ориентируясь на которую он мечтал создать свое «регулярное» государство – четкое, послушное, сильное в военном отношении, слаженно и добросовестно работающее. А гвардия боготворила своего создателя. И недаром. Дело было не только в почестях и привилегиях. Петр сумел внушить семеновцам и преображенцам ощущение участия в строительстве нового государства. Гвардеец не только был, но и осознавал себя государственным человеком. И это совершенно новое для рядового русского человека самоощущение давало петровскому гвардейцу необыкновенные силы.

\hfill

Важно и то, что у гвардейцев было преувеличенное представление о своей роли в жизни двора, столицы, России. \textbf{Петр I создал силу, на протяжении XVIII века выступающую главным вершителем судеб монархов и претендентов на престол}. Гвардейские полки, дворянские по составу, \textbf{являлись ближайшей опорой трона}. Они представляли ту реальную вооруженную силу при дворе, которая могла содействовать и возведению на престол, и низложению царей. Поэтому \textbf{правители всячески старались заручиться поддержкой гвардии, осыпали ее знаками внимания и милостями}. Между гвардией и монархом устанавливались особые отношения: гвардейская казарма и царский дворец оказывались тесно связанными друг с другом. Служба в гвардии не была доходна – она требовала больших средств, но зато открывала хорошие карьерные виды, дорогу политическому честолюбию и авантюризму, столь типичному для XVIII века с его головокружительными взлетами и падениями «случайных» людей.

\hfill

И тем не менее часто оказывалось, что «свирепыми русскими янычарами» можно успешно управлять. Лестью, посулами, деньгами ловкие придворные дельцы умели направить раскаленный гвардейский поток в нужное русло. Впрочем, как обоюдоострый меч, гвардия была опасна и для тех, кто пользовался ее услугами. Императоры и первейшие вельможи нередко становились заложниками необузданной и капризной вооруженной толпы гвардейцев. И вот эту зловещую в русской истории роль гвардии проницательно понял французский посланник в Петербурге Жан Кампредон, написавший своему повелителю Людовику XV сразу же после вступления на престол Екатерины I: \textbf{«Решение гвардии здесь закон»}. И это было правдой, XVIII век вошел в русскую историю как \textbf{«век дворцовых переворотов»}. И все эти перевороты делались руками гвардейцев.

\hfill

\textbf{28 января 1725 года} гвардейцы впервые сыграли свою политическую роль в драме русской истории, сразу после смерти первого императора приведя к трону вдову Петра Великого в обход прочих наследников. Это было первое самостоятельное выступление гвардии как политической силы.

Когда в \textbf{мае 1727 года} Екатерина I опасно занемогла, для решения вопроса о преемнике собрались чины высших правительственных учреждений: Верховного тайного совета, Сената, Синода, президенты коллегий. Среди них появились и майоры гвардии, как будто гвардейские офицеры составляли особую политическую корпорацию, без содействия которой не мог быть решен такой важный вопрос. В отличие от других гвардейских корпораций – римских преторианцев, турецких янычар – \textbf{русская гвардия превращалась именно в политическую корпорацию}.

Гвардия выбирала не просто царствующую особу, она выбирала принцип. Причем выбирала гвардия не между петровской и допетровской Россией, а она делала свой выбор в январе 1725 года между двумя тенденциями политического реформирования страны – умеренного, но несомненного движения в сторону ограничения самодержавия и неизбежного при этом увеличения свободы в стране, с одной стороны, и дальнейшего развития и укрепления военно‑бюрократического государства, основанного на тотальном рабстве, – с другой. Гвардия в 1725 году выбрала второй вариант.

\subsection{Рост городов и изменения в социальной стратификации городского населения}

В \textbf{первой четверти XVIII в.} наблюдаются существенные изменения в составе и численности городского населения. \textbf{Рекрутские наборы и рост государственных повинностей вызвали временную убыль городского населения}, бежавшего, подобно крестьянам, за окраины. В то же время в таких городах, как Казань, Тула и особенно Москва, где насчитывалось около 30 мануфактур, среди населения увеличивалась прослойка \textbf{работных людей}. С развитием мануфактур связано \textbf{появление новых типов населенных пунктов}, ставших позже городами, — Екатеринбург на Урале, Петрозаводск в Карелии, Липецк в Воронежской губернии и др.

\hfill

\textbf{В 1703 г. был основан Санкт-Петербург}. Его строили в тяжелых условиях десятки тысяч солдат и крестьян, согнанных со всей страны. Новый город заселялся ремесленниками и купцами, принудительно переводившимися из других торгово-промышленных центров. От старых городов, беспорядочно застраивавшихся деревянными зданиями, Петербург отличался строгой планировкой улиц, каменными домами, мостовыми и уличным освещением. С переездом сюда царского двора в 1712 г. Петербург стал официальной столицей государства; он \textbf{являлся морским портом, «окном в Европу», культурным и торгово-промышленным центром}.

На Адмиралтейской верфи в Петербурге, крупнейшем предприятии России, было занято свыше 10 тыс. рабочих.

\hfill

\textbf{Города получили новое административное устройство в 1720 г. с образованием Главного магистрата в Петербурге и магистратов в городах}. Регламент Главного магистрата отражал изменения в социальной структуре городского населения, но оформил эти изменения по-феодальному. Он \textbf{делил жителей посада на «регулярных» граждан в составе двух гильдий, к которым были отнесены купцы и цеховые, ремесленники, и «нерегулярных», или «подлых», людей, т. е. чернорабочих и работных людей мануфактур}.

Последние представляли обездоленную массу городского населения, \textbf{лишенную права участвовать в выборах органов самоуправления}. Социальные различия резко сказывались и среди «регулярных» граждан. Общепосадские сходы, на которых происходили выборы городских органов, представляли арену ожесточенной борьбы между верхушкой нарождавшейся буржуазии и мелким ремесленным людом. \textbf{Правительство ориентировалось на богатые слои горожан, предлагая избирать в городские органы «дельных и лучших в купечестве людей»}. Таким образом, в социальном строе русского общества наряду со старыми классами-сословиями — крестьянством и дворянством — \textbf{начали складываться элементы новых классов: рабочие мануфактур (предпролетариат) и буржуазия} (мануфактуристы, верхушка ремесленников, купечество и т. п.). Последняя получила сословную организацию с весьма существенными привилегиями, отгораживавшими ее от «подлого» народа.

В результате реформ, ускоривших темпы социального, экономического и культурного развития, \textbf{Россия в значительной мере преодолела свое отставание от передовых государств Западной Европы, сильно сказывавшееся в XVII в}. Но ее \textbf{успехи следует признать относительными}. Так, численность городского населения, являющаяся одним из показателей уровня общественного разделения труда, по данным первой ревизии, достигала лишь 3%.

\subsection{Изменения в положении крестьянства. Дискуссии о возможности отмены крепостного права при Екатерине II}

Законодательство Екатерины о пространстве помещичьей власти над крепостными людьми отличается той же неопределенностью и неполнотой, как и законодательство ее предшественников. Вообще оно было \textbf{направлено в пользу землевладельцев}. Мы видели, что Елизавета в интересах заселения Сибири \textbf{законом 1760 г.} предоставила помещикам право «за предерзостные поступки» ссылать крепостных здоровых работников в Сибирь на поселение без права возврата; Екатерина \textbf{законом 1765 г.} превратила это ограниченное право ссылки на поселение в право ссылать крепостных на каторгу без всяких ограничений на какое угодно время с возвратом сосланного по желанию к прежнему владельцу.

\hfill

Далее, в XVII в. правительство принимало челобитья на землевладельцев за жестокое их обращение, производило сыски по этим жалобам и наказывало виновных. В царствование Петра был издан \textbf{ряд указов, запрещавших людям всех состояний обращаться с просьбами на высочайшее имя помимо правительственных учреждений}; эти указы подтверждались преемниками Петра. Однако правительство продолжало принимать крестьянские жалобы на помещиков от сельских обществ. Эти жалобы сильно затрудняли Сенат; в начале царствования Екатерины он предложил Екатерине меры для полного прекращения крестьянских жалоб на помещиков.

\hfill

Екатерина утвердила этот доклад, и \textbf{22 августа 1767 года}, в то самое время как депутаты Комиссий слушали статьи «Наказа» о свободе и равенстве, издан был указ, который гласил, что если кто «недозволенные на помещиков своих челобитные наипаче ее величеству в собственные руки подавать отважится», то и челобитчики и составители челобитных будут наказаны кнутом и сосланы в Нерчинск на вечные каторжные работы с зачетом сосланных землевладельцам в рекруты. Этот указ велено было читать в воскресные и праздничные дни по всем сельским церквам в продолжение месяца. Предложение Сената, утвержденное императрицей, было так составлено, что \textbf{прекращало крестьянам всякую возможность жаловаться на помещика}. 

\hfill

При Екатерине \textbf{не были точно определены границы вотчинной юрисдикции}. В указе 18 октября 1770 г. было сказано, что помещик мог судить крестьян только за те проступки, которые по закону не сопровождались лишением всех прав состояния; но размер наказаний, каким мог карать за эти преступления землевладелец, не был указан. Пользуясь этим, за маловажные проступки землевладельцы карали крепостных такими наказаниями, которые полагались только за самые тяжкие уголовные преступления. В \textbf{1771 г. для прекращения неприличной публичной торговли крестьянами издан был закон, запрещавший продажу крестьян без земли} за долги помещиков с публичного торга, «с молотка». Закон \textbf{оставался без действия}, и Сенат не настаивал на его исполнении. \textbf{В 1792 г. новый указ восстановил право безземельной продажи крестьян за помещичьи долги с публичного торга} только без употребления молотка. Наконец, в жалованной грамоте дворянству 1785 г., перечисляя личные и имущественные права сословия, Екатерина \textbf{не выделила крестьян из общего состава недвижимого дворянского имущества}, т. е. молчаливо признала их составной частью сельскохозяйственного помещичьего инвентаря. Так, \textbf{помещичья власть, лишившись прежнего политического оправдания, приобрела при Екатерине более широкие юридические границы}.

\hfill

Это наиболее важные и заслуживающие внимания распоряжения Екатерины о крепостных людях. Неполнота этих распоряжений и закрепила тот взгляд на крепостных людей, который, помимо закона, даже вопреки ему, утвердился в дворянской среде в половине XVIII столетия. Этот взгляд состоял в \textbf{признании крепостных людей частной собственностью землевладельцев}. Законодательство Екатерины утвердило этот взгляд не столько тем, что оно прямо говорило, сколько тем, о чем умалчивало.

\hfill

Какие способы определения отношений крепостного населения возможны были в царствование Екатерины? Мы видели, что крепостные крестьяне были Прикрепленные к лицу землевладельца [как] вечно-обязанные государственные хлебопашцы. Закон определял их крепость к лицу, но не определил их отношений к земле, работой над которой и оплачивались государственные повинности крестьян. Можно было \textbf{тремя способами разверстать отношения крепостных крестьян к землевладельцам}: во-первых, их можно было \textbf{открепить от лица землевладельца, но при этом не прикреплять к земле}, следовательно, это было бы безземельным освобождением крестьян. О таком освобождении мечтали либеральные дворяне времен Екатерины, но такое освобождение едва ли было возможно, по крайней мере оно внесло бы совершенный хаос в народнохозяйственные отношения и, может быть, повело бы к страшной политической катастрофе.

\hfill

Можно было, с другой стороны, \textbf{открепив крепостных от лица землевладельца, прикрепить их к земле}, т. е., сделав их независимыми от господ, привязать их к земле, выкупленной казной. Это поставило бы крестьян в положение, очень близкое к тому, какое на первое время создало для них 19 февраля 1861 г.: оно превратило бы крестьян в крепких земле государственных плательщиков. В XVIII в. едва ли возможно было совершить такое освобождение, соединенное со сложной финансовой операцией выкупа земли.

\hfill

Наконец, можно было, \textbf{не открепляя крестьян от лица землевладельцев, прикрепить их к земле}, т. е. сохранить известную власть землевладельца над крестьянами, поставленными в положение прикрепленных к земле государственных хлебопашцев. Это создало бы временнообязанные отношения крестьян к землевладельцам; законодательство в таком случае должно было определить точно поземельные и личные отношения обеих сторон.

Такой способ разверстки отношений был всего удобнее, и на нем именно настаивали и Поленов и близкие к Екатерине практические люди, хорошо знавшие положение дел в селе, как, например, Петр Панин или Сиверс.

\hfill

Екатерина не избрала ни одного из этих способов, она просто \textbf{закрепила господство владельцев над крестьянами в том виде, как оно сложилось в половине XVIII в., и в некоторых отношениях даже расширила ту власть}.

Благодаря этому крепостное право при Екатерине II вступило в третий фазис своего развития, приняло третью форму. Первой формой этого права была личная зависимость крепостных от землевладельцев по договору — до указа 1646 г.; такую форму имело крепостное право до половины XVII в. По Уложению и законодательству Петра это право превратилось в потомственную зависимость крепостных от землевладельцев по закону, обусловленную обязательной службой землевладельцев. При Екатерине \textbf{крепостное право получило третью форму: оно превратилось в полную зависимость крепостных, ставших частной собственностью землевладельцев, не обусловливаемой и обязательной службой последних, которая была снята с дворянства}. Вот почему Екатерину можно назвать виновницей крепостного права не в том смысле, что она создала его, а в том, что это право при ней из колеблющегося факта, оправдываемого временными нуждами государства, превратилось в признанное законом право, ничем не оправдываемое.

\subsection{Общие итоги социокультурного развития России в 18 столетии. Рост социальной поляризации и обособленности сословий}

В XVIII столетии в России происходили радикальные изменения во всех сферах общественной жизни. \textbf{Основным содержанием Нового времени стало разрушение традиционного уклада и создание основ модернизации, проведение реформ «сверху», ускорявших европеизацию России}.

\hfill

В области социально-экономических отношений изменения выражались в \textbf{разложении феодального хозяйства, формировании в недрах патриархального уклада капиталистических отношений}.

В политической сфере основные усилия государства и власти были направлены на \textbf{европеизацию элементов политической структуры}. Однако проведение радикальных преобразований в сжатые сроки потребовало сверхконцентрации усилий, а следовательно, централизации и бюрократизации власти. \textbf{Возникло противоречие между тенденцией демократизации социально-политических институтов в соответствии с требованием времени и абсолютизацией власти в условиях форсированной модернизации}. Укрепляя свою власть, российская монархия, опиравшаяся на дворянское сословие, вынуждена была усилить крепостничество. Отсутствие цивилизованных механизмов разрешения возникших противоречий вызвало мощные выступления социальных низов.

\hfill

Значительные изменения происходили в системе ценностей российского общества. \textbf{Разрушение патриархальных отношений и, соответственно, формирование новых буржуазных, либеральных ценностей являлось ведущей тенденцией в развитии общественного сознания}. Однако она не определяла жизнь и сознание всего российского населения. Укорененная в аграрной сфере часть общества – и дворянства, и простого народа – оставалась в пространстве традиционной, патриархальной культуры. И все же в общественном сознании, несмотря на очевидные противоречия, присутствовала основа для консолидации. Важнейшую роль в этом отношении играла православная система ценностей, которая на долгие годы оставалась прочной основой общественной жизни.

\hfill

\textbf{Общественно-культурная жизнь России в XVIII в. отличалась богатством и противоречивостью содержания, разнообразием форм, направлений и стилей}. Развивались наука, книгоиздательское дело; формировалась система образования; происходило становление профессиональной художественной литературы и искусства; создавались новые социокультурные учреждения. Развитие буржуазных отношений, ценностей, с одной стороны, и сохранение дворянского уклада, сословных границ – с другой, отражали противоречивость социокультурного развития. Во второй половине века раскол между европеизированной дворянской и традиционной народной культурами постепенно стал преодолеваться. Формировалась \textbf{единая общенациональная культура.
}

\hfill

\textbf{Значительное влияние на этот процесс оказывала общественная мысль, развивавшаяся под воздействием европейского Просвещения}. Такие идеи, как «естественные права» человека, принадлежащие всем людям без различия вероисповедания, национальной принадлежности, пола и возраста, получили широкое распространение в российском обществе, особенно страдавшем от крепостнических отношений и других пережитков феодальной эпохи.

\hfill

Для наступавшего Нового времени были \textbf{характерны нарастающий динамизм, напряженность социокультурного развития, обусловленная форсированными темпами преобразований, противоречиями между «почвой» и «цивилизацией»}. Появлялись новые художественные стили и направления, «иноземное» происхождение которых не только не мешало, но, напротив, обогащало русскую национальную культуру. В центре ее внимания был Человек, почитаемый как Личность, его насущные проблемы, стремление к справедливости, к достойному существованию.

\end{document}